\documentclass[]{book}
\usepackage[utf8]{inputenc}
\usepackage{enumerate}				% Listas enumeradas
\usepackage{amsthm}					% Pacote matematico

\author{Lucas Arantes Berg}

% Teoremas
\theoremstyle{theorem}
\newtheorem{teo}{Teorema}[chapter]
\newtheorem{lema}[teo]{Lema}

\theoremstyle{definition}
\newtheorem{defi}[teo]{Definição}

\begin{document}
	
	\tableofcontents
	
	% Referências
	\chapter{Teoria dos Números} \label{cap:teoria}
	
		\begin{defi}[Terno Pitagórico]
			Um \emph{terno pitagórico} é formado por três números naturais $a$, $b$ e $c$ tais que $a^2 + b^2 = c^2$.
		\end{defi}
		
		\begin{teo}[Fermat-Wiles] \label{teo:teoFermat}
			Não existe nenhum conjunto de inteiros positivos $x$, $y$, $z$ e $n$, com $n > 2$, tais que $x^n + y^n = z^n$.
		\end{teo}
		
		\begin{proof}
			Seja $\Delta ABC$ um triângulo retângulo ...
		\end{proof}
		
		
	
	\section{Notação} \label{cap:notacao}
	
	\section{Resultados} \label{cap:resultados}
	
	Ver seção \ref{cap:notacao}.
		
\end{document}