\documentclass[]{article}

\usepackage{graphicx} 			% Imagens
\usepackage{amsmath}			% Matemático
\usepackage{amsfonts}			% Fontes
\usepackage[utf8]{inputenc}		% Acentos
\usepackage{subfigure}			% Conjuntos de figuras
\usepackage{tikz}				% Desenhar no Latex

% Comandos
\newcommand{\R}{\mathbb{R}}
% Comando necessita de um parametro obrigatorio
\newcommand{\vecs}[1]{(#1_1,\dots,#1_n)} 
% Comando tem o segundo argumento obrigatorio e o segundo eh opcional (default=n)
\newcommand{\vecx}[2][n]{(#2_1,\dots,#2_{#1})}


%opening
\title{Trabalhando com Comandos}
\author{Lucas Arantes Berg}

\begin{document}

\maketitle

\begin{abstract}
	Aqui vai o resumo.
\end{abstract}

\section{Teste de Comandos}

Seja $a \in \R$ tal que ...

Uma coordenada de vecs $\vecs \theta$.

Aqui vai outra $\vecs v$

E outra $\vecs p$

Coordenadas de vecx $\vecx{v}$ e $\vecx[i]{v}$ 

\end{document}
