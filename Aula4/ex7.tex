\documentclass[]{article}

\usepackage{amsmath}			% Matemático
\usepackage{amsfonts}			% Fontes
\usepackage[utf8]{inputenc}		% Acentos
\usepackage{subfigure}			% Conjuntos de figuras



%opening
\title{Trabalhando com Ambiente Matemático}
\author{Lucas Arantes Berg}

\newcommand{\x}{\dfrac{1}{2}}

\begin{document}

\maketitle

\begin{abstract}
	Aqui vai o resumo.
\end{abstract}

\section{Teste de Ambiente Matemático}

Existem várias formas de escrever uma expressão matemática em Latex, são elas o modo texto e o modo destaque.

\textbf{Modo texto:}

$e^{i \pi} + 1 = 0$


\textbf{Modo texto versão 2:}

\(e^{i \pi} + 1 = 0\)

\textbf{Modo destaque sem numeração:}

\begin{equation*}
	e^{i \pi} + 1 = 0
\end{equation*}

\textbf{Modo destaque sem numeração versão 2:}

\[
e^{i \pi} + 1 = 0
\]

\section{Exemplo de uso do modo destaque com numeração}

A fórmula de Euler é dada por:

\begin{equation} \label{eq:euler}
	e^{i \pi} + 1 = 0
\end{equation}

% Para equacoes utilize o comando \eqref
Ver \eqref{eq:euler}.

\newpage
\section{Símbolos}

Exemplo de parentêses:

$( \x )$

% Use \left( <expressao> \right) para que configurar o tamanho dos parenteses
$\left( \x \right)$

Exemplo de sublinhado:

$\underbrace{xxx}_A yy$

\section{Matrizes}

\[
\begin{pmatrix}
	1 & 2 & 3  \\
	-1 & 0 & 5 \\
	0 & 3 & 4
\end{pmatrix}
\]

Seja \( A = \left(\begin{smallmatrix} 0 & 1 \\ -1 & 0 \end{smallmatrix} \right) \) a matriz ...

\section{Alinhado}

\begin{align} 
	a_1 & = b_1 + c_1  \label{eq:align} \\
	a_2 & = b_2 + c_2 - d_2 - e_2 \nonumber
\end{align}

Segue a equação \eqref{eq:align} ...

\begin{gather} 
a_1 = b_1 + c_1  \label{eq:gather} \\
a_2 = b_2 + c_2 - d_2 - e_2 \nonumber
\end{gather}

\begin{multline} \label{eq:multiline}
	a + b + c + d + e + f + g \\
	 + h + i + j + k + l + m + n
\end{multline}

Segue uma equação que utiliza multiline \eqref{eq:multiline}

\end{document}
