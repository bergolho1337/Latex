\documentclass{beamer}

\usepackage{graphicx} 			% Imagens
\usepackage{amsmath}			% Matemático
\usepackage{amsfonts}			% Fontes
\usepackage[utf8]{inputenc}		% Acentos
\usepackage{float}				% Incluir figura no minipage

\title{Trabalhando com Apresentações}
\author[L. A. B.]{Lucas Arantes Berg}
\institute{Universidade Federal de Juiz de Fora}
\date{\today}
\usetheme{default}
\usecolortheme{orchid}

% Definindo cores
\setbeamercolor{cor1}{bg=blue!60!black,fg=white}
\setbeamercolor{cor2}{bg=white,fg=blue!60!black}

\begin{document}
	\begin{frame}
		\titlepage
	\end{frame}
	
	\begin{frame}
		\frametitle{Sumário}
		\tableofcontents[section,subsection]
		\section{Seção 1}
		\section{Seção 2}
		\section{Seção 3}
		\section{Testando blocos}
		\section{Testando colunas}
		\section{Testando itens}
		\section{Testando overlays}
	\end{frame}
	
	\begin{frame}
		\frametitle{Seção 1}
		Texto número 1.
	\end{frame}
	
	\begin{frame}
		\frametitle{Seção 2}
		Texto número 2.
	\end{frame}
	
	\begin{frame}
		\frametitle{Seção 3}
		Texto número 3.
	\end{frame}
	
	\begin{frame}
		\frametitle{Testando blocos}
		\begin{block}{Título do bloco}
			Texto dentro do bloco.
		\end{block}
	\end{frame}
	
	\begin{frame}
		\frametitle{Testando blocos}
		\begin{beamercolorbox}[shadow=true,rounded=true]{cor1}
			Texto dentro da caixa com a cor 1.
		\end{beamercolorbox}
		\begin{beamerboxesrounded}[lower=cor2,upper=cor1,shadow=true]{Título}
			Texto dentro da caixa com a cor 2
		\end{beamerboxesrounded}	
	\end{frame}
	
	\begin{frame}
		\frametitle{Testando colunas}
		\begin{columns}
			\begin{column}[t]{3cm}
				Texto na coluna 1.
			\end{column}
			\begin{column}[t]{3cm}
				Texto na coluna 2.
			\end{column}
			\begin{column}[t]{3cm}
				Texto na coluna 3.
			\end{column}
		\end{columns}
	\end{frame}
	
	\begin{frame}
		\frametitle{Testando itens}
		\begin{itemize}
			\item<1-> Primeira coisa.
			\item<2-> Segunda coisa.
			\item<3-> Terceira coisa.
		\end{itemize}
		% O identificador que aparece depois do \item eh para colocar cada novo item em um slide diferente (overlay)
	\end{frame}
	
	\begin{frame}
		\frametitle{Testando itens com alerta}
		\begin{itemize}
			\item<1-|alert@1> Primeira coisa.
			\item<2-|alert@2> Segunda coisa.
			\item<3-|alert@3> Terceira coisa.
		\end{itemize}
	\end{frame}
	
	\begin{frame}
		\frametitle{Testando \textit{overlay} em blocos}
		\begin{block}{Título 1}<1->
			Este é o primeiro bloco.
		\end{block}
		\begin{block}{Título 2}<2->
			Este é o segundo bloco.
		\end{block}
		\begin{block}{Título 3}<3->
			Este é o terceiro bloco.
		\end{block}
	\end{frame}
	
	\begin{frame}
		\frametitle{Testando \textit{overlay} em imagens}
		\begin{figure}
			\centering
			\includegraphics[scale=0.2]{img/smiley.jpeg}<1>
			\centering
			\includegraphics[scale=0.2]{img/sad-smiley.png}<2>
		\end{figure}
		\begin{itemize}
			\item<1> Agora estou feliz.
			\item<2> Agora estou triste.
		\end{itemize}
	\end{frame}
	
	\begin{frame}
		\frametitle{Slides com pausa}
		\begin{itemize}
			\item 2 é primo (dois divisores: 1 e 2).
			\pause
			\item 3 é primo (dois divisores: 1 e 3).
			\pause
			\item 4 não é primo ({\color{red}três} divisores: 1, 2 and 4). 
		\end{itemize}
	\end{frame}
	
	% Colocar fragile no argumento do frame para o verbatim funcionar
	\begin{frame}[fragile]
		\frametitle{Verbatim}
		\begin{verbatim}
			Texto exemplo
			  aqui.
		\end{verbatim}
	\end{frame}
	
\end{document}

